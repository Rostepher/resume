% File:   resume.tex
% Author: Ross Bayer

% Document Configuration
\documentclass[a4paper,margin,line]{resume}

% Packages
\usepackage[defblank]{paralist}
\usepackage{pdfpages}
\usepackage{anysize}
\usepackage[unicode]{hyperref}

% PDF Configuration
\hypersetup{
    pdftitle={Ross Bayer's Resume},
	pdfauthor={Ross Bayer},
	pdfborder={0 0 0},
	unicode=true
}

% Margin Configuration
\marginsize{0.375in}{1.875in}{0.375in}{0.375in}
\setdefaultitem{\footnotesize \textbullet}{}{}{}{}{}
\setdefaultleftmargin{0em}{}{}{}{}{}

% Constants
\newcommand{\email}{\href{mailto:ross.m.bayer@gmail.com}
                         {ross.m.bayer@gmail.com}}{
\newcommand{\github}{\href{https://github.com/Rostepher}
                          {github.com/Rostepher}}
\newcommand{\linkedin}{\href{https://linkedin.com/in/Rostepher}
                            {linkedin.com/in/Rostepher}}
\newcommand{\rvspace}{1.5mm} % spacing between heading and description

% Custom Commands
\newcommand{\rurl}[1]{\hfill {\footnotesize \url{#1}}}
\newcommand{\rdate}[1]{\hfill {\small #1}}
\newcommand{\rdescription}[1]{\small #1 \normalsize}
\newcommand{\ritem}[5] {
    \item[#1]                               % name
    \hfill \rdate{#2} \\*                   % date
    \hfill {\small \emph{#3}}               % sub-heading
    \strut \hfill \rurl{#4} \\*[\rvspace]    % url
    \rdescription{#5}                       % description
}
\newcommand{\remployer}{\ritem}
\newcommand{\rproject}{\ritem}
\newcommand{\rorganization}[4] {
    \item{\bf #1}                           % name
    \rurl{#2} \\*                           % url
    \hfill {\small \emph{#3}} \\*[\rvspace] % membership status
    \rdescription{#4}                       % description
}

\begin{document}
\name{\Large Ross M. Bayer}
\begin{resume}

% Contact Information
\section{\mysidestyle Contact \\ Information}
	\begin{asparablank}
		\item 715 Park Point Dr. Unit 3 \hfill \email
		\item Rochester, NY 14623       \hfill \github
        \item (757) 604-4736            \hfill \linkedin
    \end{asparablank}

% Objective
\section{\mysidestyle Objective}
	\begin{asparablank}
    	\item Seeking a 3 or 6 month co-op or internship beginning in either
            Summer 2015 or Fall 2015.
	\end{asparablank}

% Education
\section{\mysidestyle Education}
	\begin{compactdesc}
		\item[Rochester Institute of Technology] \rdate{August 2012 - Present}
		\begin{asparablank} { \small
			\item Major: B.S. Computer Science
			\item Expected Graduation: May 2017
		} \end{asparablank}
	\end{compactdesc}

% Technical Skills
\section{\mysidestyle Technical Skills}
    \begin{compactdesc}
        % Proficient Languages
        \item[Proficient Languages:]
        \begin{asparablank} {\small
            \item C, Java, Python, Ruby, Rust
        } \end{asparablank}

        % Farmiliar Languages
        \item[Farmiliar Languages:]
        \begin{asparablank} {\small
            \item Haskell, JavaScript, Scheme, Standard ML
        } \end{asparablank}

        % Operating Systems
        \item[Operating Systems:]
        \begin{asparablank} {\small
            \item Linux (Arch and Debian based), Mac OS X
        } \end{asparablank}

        % Tools
		\item[Tools:]
        \begin{asparablank} { \small
            \item Git, \LaTeX, Make, Regular Expressions, Vim
		} \end{asparablank}
	\end{compactdesc}

% Experience
\section{\mysidestyle Experience}
	\begin{asparadesc}
        \remployer {Rochester Institute of Technology}
                   {February 2015 - Present}
                   {Research Assistant}
                   {https://mlton.org}
        {
            Currently working as a research assistant for Professor Fluet on the
            \href{https://en.wikipedia.org/wiki/Standard_ML}{\bf Standard ML},
            whole-program optimizing compiler \href{https://mlton.org}{\bf MLton};
            developing a general-purpose annotation system for attaching arbitrary
            information to nodes in the Abstract Syntax Tree (AST) that the compiler
            can utilize.
        }
        \\

        \remployer {RockTenn}
                   {July 2014 - August 2014}
                   {IT Intern}
                   {https://rocktenn.com}
        {
            Worked on a {\bf JavaScript extension} for the BI tool QlikView
            that aggregated valid geolocation data from the database and mapped
            points of interest and other relevant information on an interactive
            map using the popular \href{https://openlayers.org}{\bf OpenLayers}
            library and \href{https://openstreetmap.org}{\bf Open Street Maps}.
            Also wrote {\bf Python scripts to geocode addresses} and store
            the results in the database using the
            \href{https://code.google.com/p/pyodbc/}{pyodbc} module and SQL.
	    }
    \end{asparadesc}

% Personal Projects
\section{\mysidestyle Personal Projects}
    \begin{asparadesc}
        \rproject {brainfuck}
                  {December 2014 - Present}
                  {Optimizing \href{https://en.wikipedia.org/wiki/Brainfuck}
                                   {Brainfuck} Interpreter}
                  {https://github.com/Rostepher/brainfuck}
        {
            An {\bf interpreter} for the esoteric, turing complete programming
            language \href{https://en.wikipedia.org/wiki/Brainfuck}{\bf Brainfuck}
            written in \href{https://rust-lang.org}{\bf Rust}. It utilizes a
            number of {\bf optimization strategies} to exponentially decrease
            the number of operations and subsequently the execution time of
            Brainfuck programs. It also supports emitting optimized internal
            Intermediate Representation (IR), optimized C code, or optimized
            Rust code.
        }
        \\

        \rproject {libstrcmp}
                  {June 2014 - December 2014}
                  {String Distance Metric C Library}
                  {https://github.com/Rostepher/libstrcmp}
        {
            An optimized {\bf C library} dedicated to fast implementations of
            various {\bf string distance metrics}, including the
            \href{https://en.wikipedia.org/wiki/Levenshtein_distance}
            {\bf Levenshtein distance} and the
            \href{https://en.wikipedia.org/wiki/Jaro-Winkler_distance}
            {\bf Jaro-Winkler distance}, as well as other phonetic metrics used
            in {\bf natural language processing} and {\bf fuzzy string matching}.
        }
        \\

        \rproject {mget}
                  {June 2014 - Present}
                  {Manga Scraping Ruby Utility}
                  {https://github.com/Rostepher/mget}
        {
            A Ruby utility to {\bf scrape manga chapters} from the multiple
            online sources, using the ruby gem \href{http://nokogiri.org/}
            {Nokogiri} for HTML parsing, and then package the images in a
            cbz (comic book zip) archive for easy reading on the go.
        }
    \end{asparadesc}

% Extracurricular Clubs and Activities
\section{\mysidestyle Extracurricular Clubs Activities}
	\begin{asparablank}
        \rorganization {FOSS@MAGIC}
                       {https://magic.rit.edu/foss}
                       {Active Member}
        {
            A group dedicated to Free and Open Source Software (FOSS) located
            inside the MAGIC Center, formerly the Innovation Center, that
            sponsors numerous projects, hackathons and meetings for like-minded
            students.
        }
        \\

        \rorganization{Computer Science House ({\small CSH})}
                      {https://csh.rit.edu}
                      {Alumni Memeber}
        {
            A special interest house with a focus on project-based education in
            the field of computer science and its related sub-fields.
        }
    \end{asparablank}
\end{resume}
\end{document}

