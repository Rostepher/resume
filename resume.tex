% File:   resume.tex
% Author: Ross Bayer

% Document Configuration
\documentclass[a4paper,margin,line]{resume}

% Packages
\usepackage[defblank]{paralist}
\usepackage{pdfpages}
\usepackage{anysize}
\usepackage[unicode]{hyperref}

% PDF Configuration
\hypersetup{
    pdftitle={Ross Bayer's Resume},
	pdfauthor={Ross Bayer},
	pdfborder={0 0 0},
	unicode=true
}

% Margin Configuration
\marginsize{0.375in}{1.875in}{0.375in}{0.375in}
\setdefaultitem{\footnotesize \textbullet}{}{}{}{}{}
\setdefaultleftmargin{0em}{}{}{}{}{}

% Constants
\newcommand{\rvspace}{2mm} % spacing between heading and description

% Custom Commands
\newcommand{\rurl}[1]{\hfill {\footnotesize \url{#1}}}
\newcommand{\rdate}[1]{\hfill {\small #1}}
\newcommand{\rdescription}[1]{\small #1 \normalsize}
\newcommand{\ritem}[5] {
    \item[#1]                               % name
    \hfill \rdate{#2} \\*                   % date
    \hfill {\small \emph{#3}}               % sub-heading
    \strut \hfill \rurl{#4} \\*[\rvspace]   % url
    \rdescription{#5}                       % description
}
\newcommand{\remployer}{\ritem}
\newcommand{\rproject}{\ritem}
\newcommand{\rorganization}[4] {
    \item{\bf #1}                           % name
    \rurl{#2} \\*                           % url
    \hfill {\small \emph{#3}} \\*[\rvspace] % membership status
    \rdescription{#4}                       % description
}

\begin{document}
\name{\Large Ross M. Bayer
    \hspace{81mm}
    \footnotesize ross.m.bayer@gmail.com | (585) 210-8071}
\begin{resume}

% Education
\section{\mysidestyle Education}
\begin{compactdesc}
    \item[Rochester Institute of Technology] \rdate{August 2012 - Present}
    \begin{asparablank} { \small
        \item Major: B.S./M.S. Computer Science
        \item Expected Graduation: December 2018
        \item GPA: 3.2
    } \end{asparablank}
\end{compactdesc}

% Technical Skills
\section{\mysidestyle Technical Skills}
\begin{compactdesc}
    % Proficient Languages
    \item[Proficient Languages:]
    \begin{asparablank} {\small
        \item C, Go, Java, Python, Standard ML
    } \end{asparablank}

    % Familiar Languages
    \item[Familiar Languages:]
    \begin{asparablank} {\small
        \item Haskell, JavaScript, Prolog, Rust, Ruby, Scheme
    } \end{asparablank}

    % Operating Systems
    \item[Operating Systems:]
    \begin{asparablank} {\small
        \item Linux (Arch and Debian based), Mac OS X
    } \end{asparablank}

    % Tools
    \item[Tools:]
    \begin{asparablank} { \small
        \item Git, \LaTeX, Make, CMake, Regular Expressions, Vim
    } \end{asparablank}
\end{compactdesc}

% Experience
\section{\mysidestyle Experience}
\begin{asparadesc}
    \remployer {Rochester Institute of Technology}
               {February 2016 - Present}
               {Reserach Assistant}
               {https://mlton.org}
    {
        Re-implementing MLton's build system to support modern architectures and
        techniques and improving overall build quality. Switching from GNU Make
        and small number of handwritten build scripts to a CMake build system.
    }
    \\

    \remployer {Hudl}
               {June 2015 - December 2015}
               {Software Development Intern}
               {https://hudl.com}
    {
        Sole developer on a mission critical application in Hudl's alerting
        pipeline. Created two open-source Go packages to interface with PagerDuty
        and Postmark's RESTful APIs.
    }
    \\

    \remployer {Rochester Institute of Technology}
               {February 2015 - May 2015}
               {Research Assistant}
               {https://mlton.org}
    {
        Learning about the general structure and implementation strategies
        of compilers.\\
        Implementing a general-purpose annotation system for the
        \href{https://mlton.org}{\bf MLton compiler}.
    }
    \\
\end{asparadesc}

% Personal Projects
\section{\mysidestyle Personal Projects}
\begin{asparadesc}
    \rproject {adhoc-prologmorphism}
              {February 2016 - Present}
              {Independent Study on Adhoc-Polymorphism}
              {https://github.com/Rostepher/adhoc-prologmorphism}
    {
        A collection of documents and interpreters from my independent study on
        Adhoc-Polymorphism in Hindley-Milner type systems.
    }
    \\

    \rproject {brainfuck}
              {December 2014 - Present}
              {Optimizing \href{https://en.wikipedia.org/wiki/Brainfuck}
                               {Brainfuck} Interpreter}
              {https://github.com/Rostepher/brainfuck}
    {
        An {\bf interpreter} for the esoteric, turing complete programming
        language \href{https://en.wikipedia.org/wiki/Brainfuck}{\bf Brainfuck}
        written in \href{https://rust-lang.org}{\bf Rust}, utilizing a
        number of {\bf optimization strategies} to exponentially decrease
        the number of operations and subsequently the execution time of
        \href{https://en.wikipedia.org/wiki/Brainfuck}{Brainfuck} programs.
    }
    \\

    \rproject {libstrcmp}
              {June 2014 - December 2014}
              {String Distance Metric C Library}
              {https://github.com/Rostepher/libstrcmp}
    {
        An optimized {\bf C library} dedicated to fast implementations of
        various {\bf string distance} and {\bf phonetic metrics} used
        in {\bf natural language processing} and {\bf fuzzy string matching}.
    }
    \\

    \rproject {mget}
              {June 2014}
              {Manga Scraping Ruby Utility}
              {https://github.com/Rostepher/mget}
    {
        A Ruby utility to {\bf scrape manga chapters} from online sources
        and then packages each chapter in a cbz (comic book zip) archive.
    }
\end{asparadesc}

% Extracurricular Clubs and Activities
\section{\mysidestyle Extracurricular Clubs \& Activities}
\begin{asparablank}
    \rorganization {Computer Science House ({\small CSH})}
                   {https://csh.rit.edu}
                   {Alumni Memeber}
    {
        A special interest house with a focus on project-based education in
        the field of computer science and its related sub-fields.
    }
\end{asparablank}

\end{resume}
\end{document}

